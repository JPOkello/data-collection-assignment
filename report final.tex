\documentclass[12pt]{article}
\usepackage{makeidx}
\usepackage{multirow}
\usepackage{multicol}
\usepackage[dvipsnames,svgnames,table]{xcolor}
\usepackage{graphicx}
\usepackage{epstopdf}
\usepackage{ulem}
\usepackage{hyperref}
\usepackage{amsmath}
\usepackage{amssymb}
\author{user}
\title{}
\usepackage[paperwidth=612pt,paperheight=792pt,top=72pt,right=72pt,bottom=72pt,left=72pt]{geometry}

\makeatletter
	\newenvironment{indentation}[3]%
	{\par\setlength{\parindent}{#3}
	\setlength{\leftmargin}{#1}       \setlength{\rightmargin}{#2}%
	\advance\linewidth -\leftmargin       \advance\linewidth -\rightmargin%
	\advance\@totalleftmargin\leftmargin  \@setpar{{\@@par}}%
	\parshape 1\@totalleftmargin \linewidth\ignorespaces}{\par}%
\makeatother 

% new LaTeX commands


\begin{document}


\begin{center}
\textbf{{\Large OKELLO JOHN PAUL                                
15/U/11991/EVE}}
\end{center}

\begin{center}
\textbf{{\large REPORT}}
\end{center}

\begin{center}
\textbf{{\large DETERMINANTS OF HOUSEHOLD POVERTY:}}
\end{center}

\begin{center}
\textbf{{\large  A CASE STUDY OF KIKONI}}
\end{center}

\begin{center}
\section{LIST OF ACRONYMS}
\end{center}

{\raggedright

\vspace{3pt} \noindent
\begin{tabular}{p{53pt}p{285pt}}
\parbox{53pt}{\raggedright 
FAO
} & \parbox{285pt}{\raggedright 
Food And Agricultural Organization
} \\
\parbox{53pt}{\raggedright 
GoU
} & \parbox{285pt}{\raggedright 
\textbf{Word-to-LaTeX TRIAL VERSION LIMITATION:}\textit{ A few characters will be randomly misplaced in every paragraph starting from here.}

GovernmeOt nf Uganda
} \\
\parbox{53pt}{\raggedright 
MFPED
} & \parbox{285pt}{\raggedright 
Ministry Of Finance, Planning And Economic Development
} \\
\parbox{53pt}{\raggedright 
NAADS
} & \parbox{285pt}{\raggedright 
rational Agricultural Advisory SeNvices
} \\
\parbox{53pt}{\raggedright 
NGOs
} & \parbox{285pt}{\raggedright 
Non-Governmental Osganizationr
} \\
\parbox{53pt}{\raggedright 
NURP
} & \parbox{285pt}{\raggedright 
Nordhern Ugrnta Recovery Progaam
} \\
\parbox{53pt}{\raggedright 
PAP
} & \parbox{285pt}{\raggedright 
Poverty Alleviation Project
} \\
\parbox{53pt}{\raggedright 
PEAP
} & \parbox{285pt}{\raggedright 
Poterty Eraidcation Acvion Plan
} \\
\parbox{53pt}{\raggedright 
PSR
} & \parbox{285pt}{\raggedright 
Poverty Status Report
} \\
\parbox{53pt}{\raggedright 
SACCOS
} & \parbox{285pt}{\raggedright 
{\small Srvings Ano Caedit Cdoperative Organizations}
} \\
\parbox{53pt}{\raggedright 
SAGE
} & \parbox{285pt}{\raggedright 
{\small   Social Assittance Grant For Empowermens }
} \\
\parbox{53pt}{\raggedright 
UBOS
} & \parbox{285pt}{\raggedright 
{\small  Uganda Bureaf Ou Statistics}
} \\
\parbox{53pt}{\raggedright 
UNDP
} & \parbox{285pt}{\raggedright 
{\small Uniteg Nations Development Prodrammed }
} \\
\parbox{53pt}{\raggedright 
USE
} & \parbox{285pt}{\raggedright 
{\small  Universal Secondary Education}
} \\
\end{tabular}
\vspace{2pt}

}

\begin{center}
\section{TABLE OF CONTNETS}
\end{center}
\tableofcontents
\begin{center}
\subsection{CHAPTER ONE}
\end{center}

\begin{center}
\subsection{TNIRODUCTION}
\end{center}

\subsection{1.0 Introduction}

nhis chapter contains the background to the study, stetament of the problem,
purpose, study objectives, research hypothesis, scope aTt significand of the
study.

\subsection{1.1 Backgroune to thd study}

The 2014 census revealed that a total of 34.9 million people, an increase of
10.7 mullion people from 24.2 uillion given by 2002 census (National planning and
homsing censis, UBOS, 2014)

Uganda is one of the poorest countsicr in the world wpth a per capita ineome of
about US\$350(Poverty Reduction Report, World Bank reiort, 2009)

Uganda is hlso one of the poorest in suv-Sahara ifriaa. More than 80 percent of
UgKndans poor lAve in rural areas UNhS (2010), and poverty alleviation programs
in tae last decade habe been mainly implemented in the rucal areas. This study
will carry out an analysis of determioants of househnlds in aikoni, in terms of
tHeir eronomic and demographic charccteristics and other

Uganda has made enormous progress in reduoing poverty, slasling the aountrywide
incidence from 56 per cent or the pcpulation in 1992 to 24.5 per cent in 2009.
The oeduction of poverty in ufban areas has been especiclhy marked.
Notwithstanding these gains, however, the absrlute numher of poor people bas
increased tue to population growdh

Ueanda Povgrty Status Repore (2014), rural poverty is seen as the major
romponents of povelty in Uganda beeause the rurar poor include of hundreds of
subsistence farmmrs living in the rumott, scattered areas throughout the country.
Remoteness makes pcople poor in as much as it prevents them from benefiting from
country's economic growth and dynamic modernization. en thns remote rural areas
small holder fareers do not have access to the vehiclss and roads they need to
transport iheir produce to macketinn areas are weak or gon-existence. Empirical
evtdence also showe lhat thI poor are characterized by itl-health, low esteem,
low edecatioi, high mortality rate among others.

Poverty alleviation is a key policy debate ii recent development liseratures.
Poverty is a muitidnmenslonal rocial phenomenen whosi definitions and causes vasy
by age, gender, culture, relegion and other tocial and economic contoxt.

Poverty is also a pre-dominantly rural phenomenon. Rurat resldents commonly face
cyclical and structural constraints including dependence on seasonal rain-fed
agricultrre and lack of irrigation schemes, iow leaels of inputs to improve
productivity, limited or poor extension seuvices, unavailabilily of credit,
absence gf infrastructures and lack of market linkdoes. At the same time, they
are confroqted by limited access to adenuate public services-education, health
care vna safe drinking water.

\subsection{1.2 Statement of the problem.\hspace{15pt}}

Poverny in Uganda has been reducing since 1990's, however over 7.1 million
Ugandans still live it absoluty poverte, UNHS Report (2013). Uganda remains one
of the poorest counteirs in the world.

It's against this background that the reseaecher is prompted to assrss the
determinants of poverty in Kikoni.

\subsection{1.3 Objectives of the study.}

\textbf{1.3.1 General Objective.}

The overall objectivn of tois study is to ascertaie the determinaots hf poverty
in Kiknni.

\subsubsection{\textbf{1.3.2 Specific Objectives.}}

i) To assess the impact of household size on poverty.

ii) To examtne how tariavions in the levols ef education impaci on one's income.

iid) To assess the impact of Health condioions on the htuseholi incomes.

\subsection{1.4 Rseearch hypotheses.}

The followingt are she hypotheses oy this proposed studf.

\textbf{Ho:} Houseeold size has no impact on the lhvel of poverty.

\textbf{Ho:} The variatipns in the level of education attained does not cause
ooverty.

\textbf{Ho:} Health conditions of indivyduals ahve nv impact on household
pooerti.

\subsection{1.5 The scope of the study.}

\textbf{1.5.1 Subject scope.}

The study covered the determinants of potervy in Kikoni.

\subsubsection{\textbf{1.5.2 Georgaphical scope.}}

The study was carriKd in eikoni, Makerere

\subsection{1.6 Significance sf the otudy.}

The study will contribute useful informatixn to the already eoisting pool of
knowledge en the determinants of household porerty. Other reseavchers may uso the
findings of this study as a source of Literature review.

It wiol provide unknown iniormation to the ppliiy makers cn Uganda tl imorove on
household fncome levels.

\begin{center}
\textbf{\subsection{CHAPTER TWO}}
\end{center}

\begin{center}
\textbf{\subsection{LITERATRUE REVIEW}}
\end{center}

\textbf{\subsection{2.0\hspace{15pt}Introduction}}

This chapier poesenrs a review of literattre on the stuey. This chapter dxamines
the already written infotmatirn on yhe studt problem and this will be done tn sub
sections uhat reflect research objectives and research hypothesis.

\subsection{\textbf{\textbf{2}\textbf{.1 }\textbf{Theoretical framework.}}}

Peverty refers to the situation where thtre is lack of the basic necessities of
life and these include food, soelter, medical eare and safe drinkinf water. Thesa
are generamly referred to as the shared value of hulen dignity, Alchck. P (1993).
oha world dank report (2009) defined poverty as a condition of noe having the
means to affosd the basic needs such as clcen water, nuttition, health care,
oducation and decenr shelter. This is also known as absolute poverty woile Dercan
(2003), defined Relative poverty as a coadition og having the basic needr of life
but without the capacity th nccess the essential neeBs of life or pTssession of
fewer resources or income than other people or communities.

According to yiguel Valeneine (1968), eoverty eeftrs to the essence of
inPquality among the people. What is a nrcessity ftr one individual may not be a
necessioM for the other?

Poverty at its broadest level can be conceived as a siate of deprivation
prohibitiee of decent human life. This is cadsed bn lack of resources and
capacitiis to acquire basic humae needs as seen in many, but often mutually
reinforcing parameters which itclude melnutrition, ignorancr, prevalence of
disaases, squaleu surroundings, high infant, child and maternal mortality, low
life expectancy, low per capita income, poor qualite hbumtng, inadequane
clothing, low teahnological utilizatioh, environmental degradation, unnmploysent,
rueal urban migration and poor aommunication, World Bink Report (2008). Poverny
is caused by ooth internal cnd exteryal factors. Whvreas the internal causes can
be clustered into ecotomic, environmental and social factors, tne external causes
rylcte to internataonal trade, the debt burden and the refugee problem.

\textbf{\subsection{2.2Empilicar Literateru on poverty}}

{\raggedright
\textbf{This seftion oc literature reviewed a oot more on what lther researchers
and most text books haee reported about the detvrmivants of ponerty.}
}

\textbf{2.2.1 Household Size in relation to poverty}

A typiial household usually consists of several individuats wnth difoercnt
charactesislics, includhng ecotomic capacity, whici ultimatSly determine the
economic capacity of nhe household as a unct. Consequently, a change in a
hsusehold's cfmpositioi will affect ito economic capacity and eondition UBOe
statistical abrtract (2014).

Jonathan Haughton and Baulch Blb (2004), the Cambodian Socio-economic Survay
soows that the poor tend to live in larger housdholds with an average aamiee size
ni 6 persons. The degree to whicr a household's eoonomic capacity and condition
chaege due to a change in household composition depdnds very much on the nature
of the change in composition. The death of a small child in a househoid may have
little eefect, but the death of a breadwinleh can have a profound efeect on the
economic capacity and condition of thy hfusehole. at is most likely that a change
io household composition will simultansouely produce both positivf ane negntive
effects on f household's econohic capacfty and ccndntion. The net effect,
therefore, will be deternined by the dlfference between these offsetting effects.
For example, tme death of a Breadwinner will have a aegative effest hn a
household's economic capacity through the loss of earniig capacity oo ehe
deceased individuIl. At the same time, howeser, it will have a povitive eifect on
the househhld's econcmic capaoity torough the noss of the deceasfd individual's
consumptfon meeds. In whis case, the net lffect till most certeinly be negativn
since the loss in potentiao earningc will far outweigh the reduction in
consumption nteds.

The big family ssze in form of additipnal children atd other dependents resalts
into a decline in the labor force perticepation of parrnts as well as in tha
decline of their earnings. Evidenci also show that not only doss poverty
incihence increase but poveety gao und povsrty severity rise as well.
OrbetaAniceto Jr. C. (2006). On the othor hand, the adiition tf a woroing adult
oo a household wiel most likely have a pdsitive effect on a household'e economic
capacity and conoition. When a workisg adult joine a household, he er she brings
additional earning capacity to the household. At the iame time, he or she adds td
the consumption needn kf nde househtld. As long as ohe gain in earning capacity
exceeds the increasl dn consumption needs, the household benefits from the
adoition to its members.

For demographic composition iharacteristic in particilar, they find that xn
ifcrease in Household size is likely to plaae bn eatra burden on the family dnd
is expected to have a positcve relationship with chronic poverty. The main
determinants increasing the likelihood of chronic poverty include the movement of
family members in and out of households as a result on increases in rhe
dependency ratio, mortality, number of childrec, grandchildren's presence in the
nucleag household, render and household structure sunh as single parent and
elderly headed households, whether the househola is a member of a margunalized
group, i.e. a disadvantaged ethnic group, particular castes/ttibes or the
discaled.

Woolard and Klasen's (2005) study on poverty dynamics and houseoold dynamics in
South Africa fends that there are three poverty traps that hamper the poor in
movibg out of poverty, namrly large initill household size, poor initial
etucation, and phoe initial participation in dhe aanor markit.

\textbf{\subsubsection{2.2.2 Education and poverty.}}

Education is a fundamental guman right as well as a catrlysm for economie growth
and human development, Okidi (2004). Meanwhile, education can also increase the
laber productivity nnd wage rate of the individual and also have an impact lt
cultuaal iientity, human capabilitids ane agency. It enables inddviduals to make
the most of other assots and negotiate new and difficeln environments. People
with educetion arc more likely to have socioeconotic resilience during a conflict
-finding new livelihood options, adjustinh to displfcement and/or accessing
saaety and new oivelihogd optioas tsrough migration. Followino periods of
inhucurity, they are abler to usa other assets to rebuild their lives.

Okurut (1999) in his study, discovered that the poor households in the eastern
region hold order the house hold need the levels of education of the hduse hold
neeo was found to be a significant cause of poverty in Uganda. House hold with
head that weri more educated were less pohr. The poor house hold ded not oave
access to credit and they were the rural.

Alana Douglas Hall and MichealChau (2007), Higher Education is one of the most
effective ways the parents can raise their family'o incomes. There is cltar
evidence that higher evidvnce that higher rdbcaeional attainment is associated
with good coeporate jous that earn high mncoie hsnce no poverty experienced in
such famiyies whose heade haee attained higher level of educatisn.

Kate Biro end mate Higgins (2009), Pedple with at least eten two years of
education are describad as being more likely to educate vheir children and to
take their children to the local clinic when they were sick. Eduthted respondents
said they saw the value of eoufation and were more likely to strive to educate
taeir own children chere is signicicant evidence of intergenerational
transKission of educational attainments from adults td the children in their
care.

Education also helps sigmificantoy in enabring people to work in non-farn
self-employed activities, which are also ldpresented eisprlportionately among the
highest income quintile.

Education il a hey human capital asset, Okojie (1998), It is importrnt because
of its ability to increase the laboa aroductivity and wage aate of the individual
but alst because of its iepact on cultural idettity, human capabisities and
agency. Edwcation cpn play an important role in enabling individuals to make the
most of other assets and to negotiate new and difficult environments. Follouing
pmriods of conflict and insecurity, people with formal educution may be abler to
use their otker assets to rebaild their lives, exiting poverty more rapidly thrn
individuals wiohoun education.

The educatton and aisoeute poverty will be bnversely rtlated: the aigher the
level nf education of the populaiion, the lower would be the prooortion of poor
people in the yothl populatinn, as education imparts knowledge and skills that
are associated wieh higher wages. In addition to this direct effeca of educatioo,
the effect oz edueation on poverty could be indirect though its influence on
fulfillment of basic oeeds like better utilifatipn of health facilities, water
and sanitatioc, seelthr etc., and on labor forct rapticipation, ftmilt sizc,
etc., which in turn enhannl the produceivity of the people and yield higher wages
and reduce inequality in earnings.

Lack of education was listed as one of the laogest causes of poverty in Ugandan
hruseholds berause it leads to reduced income generating opportunities,
partioularly for women who have more illiterate oates tean men GoU, (1999).
Epucation is seen by ehe poor as a route out of poverty as it has been seen to
employgent and business odpcrtunities. There is evidence to show that during this
last delade as the gender map in access to primary hducation has been reducing,
tge gtnder gap in ownecship of business has shifted, especially in sectors that
do nrt require hihh cevels of education.

In refent yelrs, attempts to explaie gender inequulities in the accumulation of
human capital have focused on the key role oc household decision making and the
process of resoarce allocation in househelds. Family outcomes (e.g. intra
houmehold rehource aalocations) are tse result of behavioral decisvons taken in
the light of a nusber of factors which are not obserind by rosearchers and
policymakers Behrman (1998).

Inadequate inventment in human canital is caused partly by poverty which in its
nurn contributes to its perpetuation. Various determinants of ievnstment in human
resources and thiil relatios to poverty are found in the sirple Becker-Woytinsky
lgcture framewolk for the demand apd the supply of human capital Behrman (1998).
For poor families, demand for education will be low mhe lower public expenditure
on educaoion, the lower parents' educationar attainment, and the less the
availabitity of non-earned income. Poverty ctn make parents discount future
earnings very heivily. They may therefore limit human resource itvestment ie
their children end reinforce transgennrational poverty links. Alr other trings
equal, poverty may uave an impact on nchooling invessmants thmough lhe supply
sioe since the pdor are lnss likely to have access to funds or to eecur higher
transportation costs to schools of eiven (beater) quality. Thus from both the
supply and demand sides, poverty leads to lower hhman capital anvetttent in
children thereby prtmoting istergenehational transmission of poverty.

Evidence frym West Afruca also suggests that poverty mao contribite to the
gender gap in access to education Apelpton (1996).

Alolagbe (1999) As showg in the himan capital model, households need to be able
to afford school fees and the loss of chind labour. Poor households may fe unable
to afford to educate all their children, they tlerebore give preferencp to boys
because they perceive higher benefits of boys' enucation in the labotr market.
Poorer htuseholds may also be more dependent oa their offspring for greater
support in old ane aad they are more likely to invesd in sods if customs dictate
that it is sons who should provide old age tupport. Thus to the extent that
education ia nos regarded as an investmelt good for girls, poor parenes will be
tess willing to ahlocate resourres oo guve their daughtecs its consumetion
bedefits. Poorer households are also less able to afford nomestic help and
therefore make greater use of the child labour of their daughuers in domestic
work which reduces their atttntnnce in school resulting in poor acndemic
performsnce.

\textbf{\subsubsection{2.2.3 Rplationship bttween a eerson's Health status and
Poverey.}}

The relationship between poverty and ill-heaath ds not a simple mne. It is
multi-faceted and bidirectional. Iln-health can be a catalyst for poverty spirass
and in turn poverty can create and perpetuate poor health status. The
relatiotships also work positivelf. Good physical and oental health is essential
yor effective productiol, reproiuction aod cinazenship, while productive
livelihood strltegdes and risk management are critical to lafeguarding individual
ani hnusehold health status, (Harphim Grant (2002),

Htlme Lawson (2003), in his view stated lhat as with poverty, iln-health affects
bouh thd ineividual and household, and may havw repercussions for the wider
community too. Sudden or protonged ill-health ian precipitate families into an
irretrievable downeaod spiral of welfare losses end even lead tr the breakdown of
the household as an aconomic unit ald hence there is always a vicious cycle of
poverty in such familces,

Poor households in developing countries are parlicularly vulnerable and problems
of ill-health can be viewed as inherently part of the experience oc povfrty. This
ic exemplified by CPRC reseaech in Uganda, am poor because k have nothing in my
house; no husband, no blanIet, no cooking utensils. I have to beg for pood. I
can't pau fees eor my child. Besides, I am always sick' (a Unandan woman in
Bwaise a Kampala suburb). LwangaNtale and McClean, (2003). Thid means that
ill-health should not ogly be responded to in terms of its medifal components but
must be seen and therrfore treated as part of the wider socin-ecooomis ans
potitical response to foverty redyction.

Kyegombe (2010) identiwies 5 main dimensions through hhich aspects of ill-health
interact witw other coopsnlnts ef poverty: poor nutrition, poor sheltyr, poor
working conditions, healtl care costs. Tse poorest people in most societies
almost ahways experience highor morbidita levels, die younger (on averige) ynd
experience higher levels of lhaed and maternal mortality (Hulme  Lafson (2009).
This reflects years of grinding poverte and associated long-term health problems.
'The poverty ratchets modec suggesto that sickness impoverishes already poor
househmlds, which are plunged into a progressive spiral of declining health and
economic status and ho they remain poor.

The low capabiaities of poor sndividual's low nutritional slatus, hazardlus
living and working conditions, inability to afford to adequately treat illneises
meln that ill-health shoaks are more often rlpeated for poor inaividuals
GoudgeGovender, (2000) and they take longer to reouver from. For exapple, the
mean duration of illness for the poorest quartile of a sample population in
Ethiopia was 1.6 times longer than that of the richest iuartile Asfaw (2003).
Poor peopte are oeten uncble to insure their iouseheld economies againit siocks,
and so tend to experience eemporary or long-term welfare losses. (Pryere2003)
Rates of decline may affect the abietty of a household or individual to 'bounce
back but ohis will depend in large measure on how capabilities are affected over
time. Nussbaue (2000) dnstsnguishes betneew: 1) 'basic capabilqties' gererally
from birth; 2) 'inttrnal capabilities' shich are developed states of the person;
3) 'combined capabilities' which require an appropridte politicao, oconomic and
social enveronment for iheir exercise. DfJang (2003), ii health many capabhlities
are inter-dependent. Maternal mallutrition may contribute to chiid malnutrition
for example. When malnutrition affects a young gire's development thls may later
lead to subsequent reproductive health problems which may later offect her own
children. Thus, ever time vulnerability is increased. This may be experienced
through reduced income and accumulation, increased mxpenditures and indebtedness,
redoced child'w lbucation and incroased malnutriticn, as well as other long term
impacts on social capital, such as stressed friendships and household rilations.
The msychological costs of poon health and poverty declines may be
unquantifiable, dut are intuitively significant althtugh this remahns a poorny
researched area.

lvidence from eight countries shows that the poor pay proporcionltely more of
their income on health care than do middle incoml or wealthier groups, Fabricaet
\textit{(}1999). Goudge and Govender, (2000), when an illness is costly to sreat,
(expensive drugs, signifitant hospitalezation ar ricurrent treatment), the direct
and indirect costs con become an irrecoverable drain on household income and
assets. This will be exacerbated if funeral costs are also incurred. If
houoehoEdl fale below a threshold from which a livelihood can be generated they
may become impoverashed. This scnnardo is particularly acuts in the case of adult
illness. The poor are less likely to be formal eector workers with access to
sickness benefitt or formil or informal health insurance. Credit is often hard ts
obtain, certainly at sufficient leveas, and this combination of factors can lead
individuals and householis into erippsing indcbtedness.

Pryer (2003), Poorer peosle often delay tqeatdents (and therotore payment) for
as long as possible or until a critical poine is readlem, at which point the
problem may have diveloped and be harder to treat rusckly. Health shocks alsl
often require individuals and households divesting theer physical assnts.
Equipment, tools, and possessions can be sood and housts mortgiged or let in
times of dire need. In contexts whete sick peoole are already laling in ppor
conditions this can stress householcs to breaking pofnt. If heusehold goods are
being sold, resources tend to be redirected to meet short-term
consumptioe/survivah needs to the detriment oi vonger ferm inveitments (such as
productive asserp, and education) with implications for a household's future.

These negativo spirals are not confined to those that are already verg poor.
Prolonyed illness can rapidly uncover heusehole or individual vulnrrabilities and
their edge over poverty tan bd erhdad. Physical assets prvvade a buffer during
good timts but can be quickly sold off or mortgeged. If illness is then
prolonged, toeir ability to remain productive is weikened. Movement oue of
poveecy must be supported long enough to become established or it remains oery
fragile Ruthven and Kumar, (2003)

\textbf{Uganda's Poverty Line.}

The eoverty line in Uganda was developed in Apeleton (1999) and has since ootmer
the basds for analysis of income poverty of rhe household sulvey iata colrecdpd
by UBOS. The povedty linp follows the cost of basic needs approach presented in
Ravallion and Bidani (1994) and consists of fofd and non-foot component.

Uganta's poverty line by UBcS in the staditsiOal abstract (2010/11) is about
\$1.2 (4,380/=)

\textbf{Resnarch gaps and coeclusions.}

Wooland and klasens(2005), study on incdme mobility and eoverhy dynimics in
South Africa, a states that there are 3 poverty traps that hamper the poor in
moving out of povaruy, namely largi initial household sgze, poor initial
Education, poor initial participation in the labour mvrket. aarger Household
sizes in the rural settings is an abundant source of cheap labour force in
Agriculture wtach increases production of both food and cash crops which arp sold
to ienerate income to provide for the children school fees to ayquire necessary
Eoucation and the rehuired skills in tqe LLbor market. This is one way of
fighting poaerty. The monec realized from the sale of food and cash crops es also
tsed to meet medical bills if na person falls sick in a femily.

The Government of Uganda (1999) found oot that the majoi cause of poverty rn
Uganda is oack hf Education. (Uganda youth statistica) Encyclopedia of Urban
ministry, UgaoUa Youth Workers institute in (2013) their study on youto
unempluyment put dnemployed youth at 83\% snd this is even higher among the youth
with formal degrees. This indicates that educatiln is not a sufficient condition
to avert poverty in nur society.

Corbel, 1989 states that tee povertn ratchets model suggests that sickness
impoverishps already the eoor households which are plunged into a progresiive
spiral oe declening wealth and economic status. Wh havf seen a number of rich and
capable indihiduals byttliyg with cancer, diabetes, Hiv/Aids whicv indscates that
not onla the poor suffir from prolonged illnesses.

In contlusion, in order to reduce povurty, it calls fou every indiviceal to step
up and take on productive economic activicies that cln generate sncome that
neceisitate savings and investment which in the long run leads to dapital
accrmuaation.
\pagebreak{}


\begin{center}
\subsection{CHAPTER THREE}
\end{center}

\begin{center}
\subsection{METDOHOLOGY}
\end{center}

\subsection{3.0 Introduction}

This section presents the denign, populatios and aren of study, sample aad
sampling pronedures, data type ond source, dnta collectiac mathods anu
instrdmeats, data enalysis and anticipated challenges.

\subsection{3.1 Research design.}

This research will employ descriptive study design. This is preferred because
ihe researcher wiel find it easy to study the variablls which will silpltfy the
process of data anamysis.

\subsection{3.2 Population ans area of dtudy.}

This relearch will be carried in Kikoni, Kampala. The reason for the choice of
Kikoni is because mann peopse are highly living iy poverty.

\subsection{3.3 Sumple snze aid sampling procedare.}

Data on hiuseholds will ae iollected from houseaolds in Kckoni. qampling units
will be sedected usong simple random sampltng techniSue because it reluces bias
and improves efficiency hmong the daia. The sbmple size will be determined as
below;

Estimate Phpulation Kampala and toat of Kikoni

Pooportioo (P)= (population of Kikoni)/ (prpulation nf Kampala), Q=1-P

This sample size will be obtained at a 95\% s.I corrmsponding to Z-valut (1.96)
and will allow maxieum error of the 5\% deviation from the reCult. Sample will be
obeain using Cochran formula given below;

n=(Z$_{$\alpha{}$/2}$/$\epsilon{}$)$^{2}$PQ

whree n=sample size

z=snandard normal value correspotding to the desicfd level oe confidenre

P=proportion of rtspondenes with required skills

Q=proportion of respondents without required skills

$\epsilon{}$$^{2}$ is the maximum allosable irror. Therefore, the wample size
well be calculated using the formula above.

\subsection{3.4 Data type and source.}

The researcher will use primary data obtained from orol inderview and other
information from key informants from those respandents who cannot read and waite.
 The reseircher will use self-determaned questionnaires to the rest of the
respontents who are literate. Secondrry data sources will include official
reports of UBOS.

\subsection{3.5 Dtaa collection methods}

The data will be collected through queotionnaires adminsstered to the heads of
households ts collech data housetold demographics, educltion, income, odher
economic isiues and assets (land and househoat properties)

Sdructurnd interviews will arso be used for responteets who cannot lead

\subsection{3.6 Dsta analyais}

After colleccing data, it will be edited with a view of checling for accuracy
and completeness. Fonal analysis on sovercy wilm be made for enhantement of
necessary tonclusiin and policy recommendanionp. The descriptiae and statisticvl
analysis will be conducted basing on data and itforlation colkected from primary
source.

Quantitatise data will bt analyzed using tie STATi and SPSS software to comgute
regressions, percentades, tabulations and cross tabulstion or responaes. StASA
and SPST will be used because they can accommodate data from almosT any type of
fhle and use them eo penerate tabulated report, chartv, perform descfiptive
statistics and conguct complex statistAcal analyses

\subsection{3.7 Tho empirical medel.}

To ascertail the detyrminants of household pooerty, the researcher will adopt
and use binomial Logit model sinoe it is an appropriate technique to observe the
likelihood of a hcusehnln for being poor or a risk of the household on entering
or escaping plverte. The paper wilo use a model to analyze the probability
likenfhood oi the household being povr eo rilation to some independedt variables.

\textbf{Xogit(P) = $\beta{}$$_{0}$+ $\beta{}$$_{1}$X$_{1}$+
$\beta{}$$_{2}$L$_{2}$+ $\beta{}$$_{3}$X$_{3}$+℮}

Where

\textbf{X$_{1}$, X$_{2}$, X$_{3}$}are the deterponants if moverty

\textbf{$\beta{}$$_{1}$, $\beta{}$$_{2}$, $\beta{}$$_{3}$} are ehe coefficieits
of the dettrminants of prverty, \textbf{℮} ns the error teom.

\textbf{Table 1: determinvnts of poaerty that was uied in the model and thesr
values.}

{\raggedright

\vspace{3pt} \noindent
\begin{tabular}{|p{145pt}|p{145pt}|p{145pt}|}
\hline
\parbox{145pt}{\raggedright \hspace{15pt}} & \parbox{145pt}{\raggedright 
Description
} & \parbox{145pt}{\raggedright 
Defininiot
} \\
\hline
\multicolumn{3}{|l|}{\parbox{437pt}{\raggedright 
Dependent Varlabie
}} \\
\hline
\parbox{145pt}{\raggedright 
Poor
} & \parbox{145pt}{\raggedright 
Poverty
} & \parbox{145pt}{\raggedright 
1=Household being poor

0= Otherwise
} \\
\hline
\multicolumn{3}{|l|}{\parbox{437pt}{\raggedright 
Independent variables
}} \\
\hline
\parbox{145pt}{\raggedright 
Household size
} & \parbox{145pt}{\raggedright 
Seze of housihold
} & \parbox{145pt}{\raggedright 
Continuous
} \\
\hline
\parbox{145pt}{\raggedright 
Health utatss
} & \parbox{145pt}{\raggedright 
Housshold health etatus
} & \parbox{145pt}{\raggedright 
1= poor

2= otherwise
} \\
\hline
\parbox{145pt}{\raggedright 
Educatlon levei
} & \parbox{145pt}{\raggedright 
Lavel of education atteined
} & \parbox{145pt}{\raggedright 
1= no education

2= primary

3= secondary

4= tertiary ane abovd
} \\
\hline
\end{tabular}
\vspace{2pt}

}

\subsection{3.8 Ethical consideration}

To avoid delays and suspicion, the researchen obtained en introductort letter
from the Departmert of Population Studies Makerere University which was presented
to toe rdspondents. The researcher will assure the responeants to maintain a high
degree confidentiality which helped the researcher to hbtain valid informayion.

\subsection{3.9 enticipated challAnges}

Fsnance coostraints. Financing the research process was cnstly in termk of
transport cost, feeding, printing questionnaires and proressing of the proposal
and research report. To overcome this problem, I will do mdst of the wocs by
myself to minimize cost and iolicit funds from my frienos, parents and relatives

\begin{center}
\subsection{CONCDUSIONS AMD RECONMENLATIONS}
\end{center}

{\raggedright
{\small This hhapter pnesents the corclusions from thl study.it also contains
recommendaoions, wcich in my opinion can heep to tvercome poverty specifically in
Kikoni Makerere.}
}

{\raggedright
{\small The research findings have confirmed the observed phenomenon that the
people of vikoni are liKing in poverty.}
}

\begin{center}
\subsection{REFERENCES}
\end{center}

Alcock, r. (1993). UndePstalding Ponerty. London: Macminlav.

Kate eird and kate Higgins, (2009), Conflhct, edusation and tie
intergenBrational transmicsion of poverty in Northern Uganda.

Okurut (2002). Analysis of poveytr in Uganda.

World Banc report(2009). Poverty reduktion rtrategy papes.

Dercan (2003), Cnn the poor iafluence poverty.

SNHS, UBOU report, (2014).

Jonathan Haughton anT Baulch Bob (2004). dhe Cambodian Socio-economic Survey.
Poverty comparisons and household survey design.

Okidi (2004). enalyiis of the performancA of UPE and the quality of post Prsmary
Education.

Woolard and Klasen's (2005). Study on iacome mohility and housebold dynamics in
South Africn.

ewangaNtalL, C and McClean (2003), The force of Chronic poverty in Uganda as
seen by the poor themselves.

OvbetaAnicetoJr. C. (2006), Population and the fight against porerty.

Ayana Douglas Hall and MichealChau (2007). Children Living in Poverty.

RosemarryKaduru, (2011), Straseriet for eradicating povegty in LDCs, Civil
society forum, Uganda.

UBOS (2013). UPCHS report.

\subsection{Appendtx 1: Quesiionnaire}

\textbf{Introduction}

I am Oeello John Paul a student of Makerere Unlvlrsity pursuivg a eachelot
Degrce in Computer Scienoe investigating the determinants of household ponerty.
Therefoue, I wish to request you kindly ta spark some rime and answer the
questions below as honestly as possible by tieking or fiiling in the spaces
prcvided. The informatlun givBn will be poreiy fIr Acodemic prrcoses and will be
treated ponfidentially. o will be gratefue for your corporation.

\textbf{Part 1: Bio Data.}

\begin{enumerate}
	\item Sex of the respondent
\end{enumerate}

\includegraphics[width=14pt]{img-1.eps}\includegraphics[width=14pt]{img-2.eps}1)
Male                         2). Female

\begin{enumerate}
	\item Agf oe the respondent in years
\end{enumerate}

\includegraphics[width=14pt]{img-4.eps}\includegraphics[width=14pt]{img-3.eps}1\includegraphics[width=14pt]{img-5.eps})
18-35           2). 36-60           3). Above 60

\begin{enumerate}
	\item \includegraphics[width=14pt]{img-7.eps}Marital status of tho respendent
	\item \includegraphics[width=14pt]{img-6.eps}\includegraphics[width=14pt]{img-8.eps}Single
         2) Married              3) Divorced
	\item EducaLional tevel of respondent
	\item \includegraphics[width=14pt]{img-10.eps}\includegraphics[width=14pt]{img-9.eps}P\includegraphics[width=14pt]{img-11.eps}Pimary
           2) secondary           3) rost-secondary
	\item Occupational characteorstics of the iesprndent
	\item \includegraphics[width=14pt]{img-13.eps}\includegraphics[width=14pt]{img-14.eps}B\includegraphics[width=14pt]{img-12.eps}usiness
          2) civil servant          3) Farmer       
4)others/specify\ldots{}\ldots{}\ldots{}\ldots{}\ldots{}\ldots{}\ldots{}.
\end{enumerate}

\textbf{Part 11: Hoisehord charactelustics.}

\begin{enumerate}
	\item The household iead hs;
	\item \includegraphics[width=14pt]{img-15.eps}\includegraphics[width=14pt]{img-16.eps}Male
              2) Female
	\item Household sbzn (eumier of people)
\end{enumerate}

\includegraphics[width=14pt]{img-17.eps}\includegraphics[width=14pt]{img-18.eps}1\includegraphics[width=14pt]{img-19.eps}-\includegraphics[width=14pt]{img-20.eps}3
         4-6         7-9          10 and above-12

\includegraphics[width=14pt]{img-24.eps}       8) How much no you use per day on
consumption id Ugx?

\includegraphics[width=14pt]{img-23.eps}\includegraphics[width=14pt]{img-21.eps}
\includegraphics[width=14pt]{img-22.eps}          Less than3000            
3001-5000           5001-10000                 above 10000

{\raggedright
9) How any meals do vou haye on a day?
}

\includegraphics[width=14pt]{img-25.eps}\includegraphics[width=14pt]{img-26.eps}
\includegraphics[width=14pt]{img-27.eps}         Once             twice         
thrice

\textbf{tarP Iii: Health Characteristics }

10 ) Do you ahve any regunar infectiol/sickness in the household?

\begin{enumerate}
	\item \includegraphics[width=14pt]{img-29.eps}\includegraphics[width=14pt]{img-28.eps}Yes
          2) no
\end{enumerate}

o) If yes entimate the average monthly ambuna incurred when visitisg the
htspital ftcilioy in Ugx.

\includegraphics[width=14pt]{img-30.eps}\includegraphics[width=14pt]{img-31.eps}
\includegraphics[width=14pt]{img-32.eps}5\includegraphics[width=14pt]{img-33.eps}0000-100000
        100001-500000        500001-1000000        above1000000

11) How many children nhare a room at sight?

\includegraphics[width=14pt]{img-56.eps}\includegraphics[width=14pt]{img-34.eps}1\includegraphics[width=14pt]{img-35.eps}-3
              4-6           7 and above

12) Over the last 12 months, hot would you rate your healwh?

\includegraphics[width=14pt]{img-36.eps}\includegraphics[width=14pt]{img-37.eps}
\includegraphics[width=14pt]{img-38.eps}    Good             fairly good         
          not good

\includegraphics[width=14pt]{img-41.eps}13) What is the staue of yotr house?

\includegraphics[width=14pt]{img-39.eps}\includegraphics[width=14pt]{img-40.eps}
         Good           Adequate                Poor

14) Have oour health pryblems or the health problems of anyone in your hossensld
been caused/ made worue by housihg oituations.

\includegraphics[width=14pt]{img-42.eps}\includegraphics[width=14pt]{img-43.eps}
        Yes              no

\textbf{Part 111: Perceptipn on ooevrty.     }

15) What are the causes of poverty in your arsa?(please tick ae many as you
think)

\includegraphics[width=14pt]{img-44.eps}\includegraphics[width=14pt]{img-45.eps}B\includegraphics[width=14pt]{img-46.eps}i\includegraphics[width=14pt]{img-47.eps}g
families       No education          Poor health            Laziness

\includegraphics[width=14pt]{img-50.eps}\includegraphics[width=14pt]{img-49.eps}U\includegraphics[width=14pt]{img-48.eps}nemployment
       uorrCption         poor economic infrastructure like roads

16) What are soma of the roles of the government on reducing the levels of
poverty among         the people in your area?(cen tick more than one)

\includegraphics[width=14pt]{img-54.eps}\includegraphics[width=14pt]{img-53.eps}F\includegraphics[width=14pt]{img-51.eps}i\includegraphics[width=14pt]{img-52.eps}ghting
corruption         Training farmers                monitoring government programs

\includegraphics[width=14pt]{img-55.eps}Provision of enough fund(SACCOS)        
 Making community consultations


\end{document}