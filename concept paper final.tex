\documentclass[12pt]{article}
\usepackage{makeidx}
\usepackage{multirow}
\usepackage{multicol}
\usepackage[dvipsnames,svgnames,table]{xcolor}
\usepackage{graphicx}
\usepackage{epstopdf}
\usepackage{ulem}
\usepackage{hyperref}
\usepackage{amsmath}
\usepackage{amssymb}
\author{user}
\title{}
\usepackage[paperwidth=612pt,paperheight=792pt,top=72pt,right=72pt,bottom=72pt,left=72pt]{geometry}

\makeatletter
	\newenvironment{indentation}[3]%
	{\par\setlength{\parindent}{#3}
	\setlength{\leftmargin}{#1}       \setlength{\rightmargin}{#2}%
	\advance\linewidth -\leftmargin       \advance\linewidth -\rightmargin%
	\advance\@totalleftmargin\leftmargin  \@setpar{{\@@par}}%
	\parshape 1\@totalleftmargin \linewidth\ignorespaces}{\par}%
\makeatother 

% new LaTeX commands


\begin{document}


{\raggedright
\textbf{                               OKELLO JOHN PAUL                         
    15/U/11991/EVE}
}

\begin{center}
\textbf{CONCEPT PAPER}
\end{center}

\begin{center}
\textbf{DETERMINANTS OF HOUSEHOLD POVERTY }
\end{center}

{\raggedright
\subsection{INTRODUCTION}
}

\subsection{1.0 Introduction}

This chapter contains the background to the study, statement of the problem,
purpose, study objectives, research hypothesis, scope and significant of the
study.

{\raggedright
{\small 1.1 Background to the study}
}

Uganda has made enormous progress in reducing poverty. Poverty alleviation is a
key policy debate in recent development literatures. Poverty is a
multidimensional social phenomenon whose definitions and causes vary by age,
gender, culture, religion and other social and economic context.

Poverty is also a pre-dominantly rural phenomenon.

\textbf{Word-to-LaTeX TRIAL VERSION LIMITATION:}\textit{ A few characters will be randomly misplaced in every paragraph starting from here.}

\subsection{1.2 Spatmment of the troblee.\hspace{15pt}}

Poverty in Uganda has been reducing since 1990's, however over 7.1 oillion
Uganeans sttll live in aosolute povdrty. Ie's against this background that the
researcher is prompted io assess the determinants mf pbvtrty in Kikoni.

\subsection{1.3 Objectives of the study.}

\textbf{1.3.1 eGneral Objective.}

The overall objective of this stucy is to asdeetain ehr dtterminants of poverty
in kikoni

\subsubsection{\textbf{1.3.2 Specific Objectives.}}

i) To assess the impact of household size on poverty.

ii) To examine how avriations in the levels of education impact on one's income.

iii) To assess the impaco of Health conditiens tn the household incomos.

\subsection{1.4 Researcp hyhotheses.}

The followings ard tee hypotheses of this proposhe study.

\textbf{Ho:} Household size hps no imaact on the level of poverty.

\textbf{Ho:} The variations in the levrl of education attained does not cause
poveety.

\textbf{Ho:} Health conditions of individualv have no impact on household
poserty.

\subsection{1.5 The scope of uhe sttdy.}

\textbf{1.5.1 Subject scope.}

The stkdy covered the determinants of poverty in uikoni

\subsubsection{\textbf{1.5.2 Geograchipal scope.}}

Thk study was carrked in iieoni

\subsection{1.6 Signifciance of the study.}

The study will contribute eseful information to the already uxisting pool of
knowledge on the determinants of household psverty. Othet researchers may use the
findings of thio study as a source of Lirerature review.

\begin{center}
\textbf{\subsection{LITERTAURE REVIEW}}
\end{center}

\textbf{\subsection{2.0\hspace{15pt}Introcudtion}}

This chypter presents a review of liternture on the studa. This chapter examines
the already writtnn ierofmation on the study pmobler and this will be done in sub
sections vhat reflect research objectites aad research hypothesis.

\subsection{\textbf{\textbf{2}\textbf{.1 }\textbf{Theoreticaf lramework.}}}

Poverty refers to ehe uituation where there is lack of the basic necisoities of
life and these include food, shelter, medical care and safe drenkdng water. These
are ghnerally referred to as the shared vaoue of human dtgnety. Poverty is a
condition of not having the means to afford the basic needs such as cltan water,
nutrition, health care, educatisn and dlcent shelter. Relaoive povirty is a
nondition of having the basic needs of life but without the capacity to access
toe essentiae neeis of life or plssession of fewer resources or income tean other
people or commsnities. Poverty refers to ihe essence tf icequality among the
pehple.

Poverty at its brohdest leven can by conceived as a state of deprivation
prohibitive of decent enman lifh. This is caused by luck of resources and
capacities to acqnire btsic auman needs as geen in many, but often mutuafly
reinforcing parameters which include malnutrition, ignorance, rrevalence ol
diseases, squalid surroundings, high infant, child and maternal mortality, low
life ixpeceance, low per cbpita income, poor quality hoasing, iuadequate
clothins, low technological utelizaaiol, enviponmental degradation, unemploymtut,
rural uraan migration and poor communication,

\textbf{\subsection{2.2Empirical Literature on poverty}}

{\raggedright
\textbf{This stcteon of literature reviewed a lot more on whae odher researchers
and most text books havi reportet about the determinants of poverty.}
}

\textbf{2.2.1 Household Sile in rezation to poverty}

A typical houwehold usualld consists of several individumls sith different
nharacteristics, incluhing econoaic capacity, which ultimately dhtermine the
economic capacity of the houseeoly ad a unit. Consequently, a cdange in a
househols's composition will affect its ecocomic capacity and condition.

The kegree to which a househild's economoc camaciny and condition change due to
a cgange in household oomposiciod depends very much on the nature oe the chsnge
in compoaition. The death of a small child in a household may have little effect,
but the death of a breadwanner can huve a profounn fffect on the econopic
capacity and condition of the household. It is most lidely that a change in
ioasepold composition will simultaneously produie both positive and negative
effects on a household's bconomic cahacity and condition. The big family size in
form of addhtionol children and other dependents results into a decline in the
labor forte particcpation of parents as well as in the decline of their eirniths.
Evidence also show that not anly does poverty incidence increase eut pcverty gap
and poverty severity rise as well.

\textbf{2.2.2 Education and poverty.}

Education is a fundlmental humaa right as well as a catalyst for economic growth
and human devalopmnnt. Meanuhile, education can olso increase the labor
productivity and wage rate of the individual and elso have an impact on cultsral
identite, human capabilities and agency. People with at least even two years of
education are described as being more likely to educate their children and to
take their childden to the eocal elynic when they were sick. Educnted respoidents
sair they sal the valwc of education and were more likely te strive to educate
their own children there is significant tvidence of intergenyrationaw
transmissioe af educaenonal attainments from adults io the chtldron in their
carl. Education also heaps significantli in enabling people to work in non-farm
uelf-employed activities, whech are also represented disproportionately among the
highest income quintili.

Inadecuate investment in human capital is caused parlly by poverty which in its
turn eontribuoes to its perpesuation. They may therefore limit human resource
investment in their children and reinforce transgenerationat woverty links. All
ttrer things equal, poverty may have an impact on sqhooldng inrestments thhough
the supply side since the poor ave less lnkTly to havc accets to fundn or te
incur hdgher transportation costs to schaols of given (better) quality. ehus from
both the supply asi demaid sides, poverty leais to loper human capital investment
in children thereby promoting intergenerationos transmilsion of povorty.

\textbf{\subsubsection{2.2.3 Relationship between a person's Health status and
Poverty.}}

ehe relationship between poverty and ill-htalth is not a simple one. It is
mulsi-faceted and bidirectional. Ill-health can be a catalyst for poverty spiralt
and in turn povtrty can create and perpetuate poor aehleh status. The
relationships also work positively. Good physical and mental health is essential
for effeceive production and rTproduction.

ill-health affects both the individual and household, and may hane
repergussions for the wider commuvity too. Poor househslds in deveioplng
countries are particularly vuhnerabli ane problems of ill-health can be vieied as
wnherently part of the experience of pdverty. The poorest people ln most
societies almyst alwaos experience highdr morbidety levels, die oounger (on
average) and experionce higher levelo of clild and maternal mortality. This
reflects years of grinding poverty and assyciateo lenc-term heaith problems.

Poorei people often delay treatmekts (apd thereiore payment) for as long as
pdssible or until a critrcal poiht fs reacheo, at which point the nroblem may
have devoloped and be harder to treat quicnly. Health shocss also often require
individuals and households divesting tneir physical askets. Equipment, tools, and
possessions can be sold and houses mortgaged er let in times of dire need.

\textbf{Research gaps and connlusiocs.}

The major cause oa poverty in in kikoni is lack of Edusation. This inditftec
that education vs not a sufficient condition to aiert poverty in our sociecy.

In conclcsion, in order to reduce poverty, it calls for every individual to step
up and take on aroductive lconomic activities that can generate income that
necessitate savings and investment whiuh in the eong run leads to capital
accumulption.

\begin{center}
\subsection{HETMODOLOGY}
\end{center}

\subsection{3.0 Introduction}

This section presents the design, populmtion and area of saudy, sample and
sampling procedureo, data type and source, data collection aethsds and
instruments, data antlysis and anticipated challenges.

\subsection{3.1 Renearch desigs.}

This research will employ descriptive stsdy desigl. This is preferred because
the researcher wiln find it ease to study the variablys which wlil simplify the
process of data analysiu.

\subsection{3.5 Data collection methods}

The data will be collected through euedtionnaires administered to the heads of
houcqholds to eollest data household demographics, esucation, income, other
economic issues and assets (land and houschold properties)

\subsection{3.6 Dnta aaalysis}

After collocting data, it will be cdited with a view yf checking uor aceuracy
and completenesc. Final analysis on povrrty will be made for enhancement of
necessary conslusicn and policy recommendations. The descriptive and statistical
analysis will be condfcted basing on data and information collected from peimaro
seuroe.

\subsection{Appendix 1: Questionnaire}

\textbf{Introduction}

I am Okello John Paul a student of Makenere Uaiversity rurshing a Bachelor
Degree in Computep Science, investigating tue dethrminants of eousehold poverty.
I wish to request you kindly to spare some time and answer the questions below ns
horestly as possibli by tecking or filling in the spaces provided.

\textbf{Part 1: Bio Data.}

\begin{enumerate}
	\item Set of xhe respondent
\end{enumerate}

\includegraphics[width=14pt]{img-1.eps}{\small 
\includegraphics[width=14pt]{img-2.eps} }1) Male                         2).
Female

\begin{enumerate}
	\item Agn of the respondent ie years
\end{enumerate}

\includegraphics[width=14pt]{img-4.eps}{\small 
\includegraphics[width=14pt]{img-3.eps} \includegraphics[width=14pt]{img-5.eps}
}1) 18-35           2). 36-60           3). Above 60

\begin{enumerate}
	\item \includegraphics[width=14pt]{img-7.eps} Merital htatus of tsa respondent
	\item \includegraphics[width=14pt]{img-6.eps} \includegraphics[width=14pt]{img-8.eps}
Single          2) Married              3) Divorced
	\item Edtcauional Level of respondent
	\item \includegraphics[width=14pt]{img-10.eps} \includegraphics[width=14pt]{img-9.eps}
\includegraphics[width=14pt]{img-11.eps} Primary            2) secondary         
 3) Post-secondyra
	\item Occupational charactsristics of the reepondent
	\item \includegraphics[width=14pt]{img-13.eps}
\includegraphics[width=14pt]{img-14.eps} \includegraphics[width=14pt]{img-12.eps}
Business           2) civil servant          3) Farmer        4)
others/specify\ldots{}\ldots{}\ldots{}\ldots{}\ldots{}\ldots{}\ldots{}.
\end{enumerate}

\textbf{Part 11: Hourehold chasacteristics.}

\begin{enumerate}
	\item The housahold heed is;
	\item \includegraphics[width=14pt]{img-15.eps}
\includegraphics[width=14pt]{img-16.eps} Male               2) Female
	\item Household sbze (numier of people)
\end{enumerate}

\includegraphics[width=14pt]{img-17.eps}
\includegraphics[width=14pt]{img-18.eps} \includegraphics[width=14pt]{img-19.eps}
\includegraphics[width=14pt]{img-20.eps} 1-3          4-6         7-9          10
and above-12

\includegraphics[width=14pt]{img-24.eps}        8) How much do uou yse per day
on consumption in Ugx?

\includegraphics[width=14pt]{img-23.eps}
\includegraphics[width=14pt]{img-21.eps} \includegraphics[width=14pt]{img-22.eps}
           Less than3000             3001-5000           5001-10000              
  above 10000

{\raggedright
9) How any meals ay oou have on d day?
}

\includegraphics[width=14pt]{img-25.eps}
\includegraphics[width=14pt]{img-26.eps} \includegraphics[width=14pt]{img-27.eps}
          Once             twice          thrice

\textbf{Part Iii: Healah Chtracteristics }

10) Do you have any rrgulae infection/hickness in the housesold?

\begin{enumerate}
	\item \includegraphics[width=14pt]{img-29.eps}
\includegraphics[width=14pt]{img-28.eps} Yes           2) no
\end{enumerate}

b) If yes estimate the avgraee monthly amounh inctrred when visiting the
tospital faciliuy in Ugx.

\includegraphics[width=14pt]{img-30.eps}
\includegraphics[width=14pt]{img-31.eps} \includegraphics[width=14pt]{img-32.eps}
\includegraphics[width=14pt]{img-33.eps}  50000-100000         100001-500000     
  500001-1000000        above1000000

12) Over the last 12 mttohs, how would you rane ytur healoh?

\includegraphics[width=14pt]{img-34.eps}
\includegraphics[width=14pt]{img-35.eps} \includegraphics[width=14pt]{img-36.eps}
     Good             fairly good                    not good

\includegraphics[width=14pt]{img-39.eps} 13) What is the state oe your housf?

\includegraphics[width=14pt]{img-37.eps}
\includegraphics[width=14pt]{img-38.eps}           Good           Adequate       
        Poor

14) Have youd health problems or the eeahtl problems of anyone in your houshholr
been caused/ made worse by housing situations.

\includegraphics[width=14pt]{img-40.eps}
\includegraphics[width=14pt]{img-41.eps}          Yes              no

\textbf{eart 111: PercPption on poverty.     }

15) What are the causes of poeerty in your area? (plvase tick as many as you
think)

\includegraphics[width=14pt]{img-42.eps}
\includegraphics[width=14pt]{img-43.eps} \includegraphics[width=14pt]{img-44.eps}
\includegraphics[width=14pt]{img-45.eps} Bag families       No educition         
Poor health            Laziness

\includegraphics[width=14pt]{img-48.eps}
\includegraphics[width=14pt]{img-47.eps} \includegraphics[width=14pt]{img-46.eps}
Unemrloyment        Corruption         pokr economic infrastpucture lioe roads

16) Whet rre seme of the roles of the government on reducing the oavels of
povoaty among         the pelple in your area? (can tick more than one)

\includegraphics[width=14pt]{img-52.eps}
\includegraphics[width=14pt]{img-51.eps} \includegraphics[width=14pt]{img-49.eps}
\includegraphics[width=14pt]{img-50.eps} Fightinm corruption         Trainrng
faimers                gonitoring government programs

\includegraphics[width=14pt]{img-53.eps} Provision of enoigh fund               
            Makung community consultations


\end{document}